\documentclass[10pt,twoside,twocolumn,openany,nodeprecatedcode,a4paper,bg=full]{dndbook}
\usepackage[english]{babel}
\usepackage[utf8]{inputenc}
\usepackage[singlelinecheck=false]{caption}
\usepackage{listings}
\usepackage{shortvrb}
\usepackage{stfloats}

\renewcommand{\familydefault}{\sfdefault}

%\captionsetup[table]{labelformat=empty,font={sf,sc,bf,},skip=0pt}

\MakeShortVerb{|}

\lstset{%
  basicstyle=\ttfamily,
  language=[LaTeX]{TeX},
  breaklines=true,
}

\title{Loose rocks}

\begin{document}
\section{Loose rocks}
\subsection{The adventure}
The characters are members of the Adventurers Guild of Snowmelt. Snowmelt is a
small town at the foot of The Whites mountain range. Life is good here. Lots of
farming, and a decent amount of trade. One day, while the characters are on
their way to the Adventurers Guild, a rider and their horse come running into
town. They have definitely seen better days. The riders' clothes are torn to
shreds, and the horse is on the verge of death.

\begin{DndReadAloud}
  As their horse collapses under them, the rider stumbles to you. They speak of
  a goblin attack on their caravan. They are the only survivor, especially now
  that the horse is also dead. As you bring the survivor inside, they are asked
  to investigate the goblin attack. Goblins do live nearby, but they've never
  dared to attack a trade caravan before. What happened?

  As you make their way along the trade route looking for the site of the
  attack, you see some broken carts and a lot of rubble in the distance. As you
  make your way there, you hear some bickering in Goblin. Eventually, one of
  the goblins steps forward and introduces the band of goblins you.

  The goblin introduces himself as Nubbs, the leader of this rag-tag band of
  goblins named Nubbs' Nubbins.
\end{DndReadAloud}

At this point, ask the players for a DC10 insight check. Those succeeding this
check notice that not all goblins seem to agree with this name.

\begin{DndReadAloud}
  Nubbs asks you to help them. You see, their house in the mountains has been
  attacked. Somehow they've been attacked. Now, you all look like strong
  adventurers, won't you help us little old goblins out?
\end{DndReadAloud}

When confronted about the attack, Nubbs will deny at first. When pressed, he'll
admit to it. But he's adamant that it was necessary for their survival. After
all, they've been forced out of their home.

If the players agree to help Nubbs, he'll lead them to their home.
If they players decide to attack the goblins, run combat as normal. Afterwards,
when they report back to the Adventurers Guild, the Guildmaster will ask them
to investigate. He'll also give them directions to where the goblins' house is.

Once there, the door will refuse to let the players in. It's reason? They're
not goblins, so they don't belong here. This is an old magical door, and there
are no other entrances as far as the players can see. One of the ways the
players could solve this is by going out of sight of the door, and then "walk
in" on their knees.

Once inside, either the goblins lead the players to the mining tunnel they
recently opened, or a trail of goblinblood will lead the players there. Next
to the entrance, there is still a big warning sign in Common and Dwarvish. This
warning sign will be spotted by the players. The sign says \textit{WARNING:
  Moving ground, keep out}. One of the goblins has seen it, but doesn't want to
admit it, afraid that others will blame him for all of this. The tunnel used to
be boarded up, but the boards have been removed and lay next to the entrance.

Once the players are inside, the ground will suddenly turn to difficult
terrain, and a deep rumble can be heard throughout the tunnel. With a DC18
perception check the players can spot where the sound is coming from. Those
who fail this check, including the Stone Elemental, are surprised for the first
round of combat. Roll for initiative for the final combat with the Stone 
Elemental. The Stone Elemental will mostly try to get the players out of it's
hair. It mostly wants to be left alone and have a calm environment.

The Stone Elemental does not necessarily need to be defeated. If the players
can convince the goblins to just leave it alone, and maybe board up the tunnel
again, that is also a conclusion of the one-shot.

\pagebreak
\subsection{Monsters and NPCs}
In this adventure there are two monsters, and a few NPCs. The goblins should be
treated as possible monsters, but more likely they'll be seen as NPCs by the
players.

\subsubsection{Monsters}
This one-shot has a few goblins (3 to 6, depending on the amount of players)
that should have a distinct personality. Pick one or roll on the personality
table below.

\begin{DndTable}[header=Goblin personality table]{ccX}
1d6 & Personality & Details \\
1 & Leader & Wants to be a leader, but doesn't know how to. Narcistic \& clumsy \\
2 & Inventor & Inventor that constantly falls short from actually inventing \\
3 & Tactician & Tactical genius that has ideas about defeating the elemental \\
4 & Animal handler & Can summon up to \DndDice{1d6} rats or wolves to help the players \\
5 & Some value & A \\
6 & Some value & A \\
\end{DndTable}

\begin{DndMonster}[float*=b,width=\textwidth + 8pt]{Stone Elemental}
  \begin{multicols}{2}
    \DndMonsterType{Large elemental, unaligned}

    % If you want to use commas in the key values, enclose the values in braces.
    \DndMonsterBasics[
      armor-class = {16 (Natural Armor)},
      hit-points  = {81 \DndDice{12d10 + 15}},
      speed       = {20 ft., burrow 30 ft.},
    ]

    \DndMonsterAbilityScores[
      str = 18,
      dex = 7,
      con = 17,
      int = 5,
      wis = 6,
      cha = 4,
    ]

    \DndMonsterDetails[
      skills = {Stealth +0},
      senses = {darkvision 60 Ft., tremorsense 60 Ft., passive Perception 8},
      languages = {---},
      challenge = 3,
    ]
    % Traits
    \DndMonsterAction{Earth Glide}
    The elemental can burrow through nonmagical, unworked earth and stone. While doing so, the elemental doesn't disturb the material it moves through.

    \DndMonsterSection{Actions}
    \DndMonsterAction{Multiattack}
    The elemental makes two slam attacks.

    \DndMonsterAttack[
      name=Slam,
      distance=melee,
      type=weapon,
      mod=+4,
      reach=10,
      targets=one target,
      dmg=\DndDice{1d8+3},
      dmg-type=bludgeoning,
    ]

    \DndMonsterAction{Ground Pound}
    The elemental stomps the ground. Each creature around the elemental within 5 feet radius must make a DC13 dexterity saving throw or fall prone.

    % Legendary Actions
    \DndMonsterSection{Legendary Actions}
    The foo can take 1 legendary action, choosing from the options below. Only one legendary action option can be used at a time and only at the end of another creature's turn. The foo regains spent legendary actions at the start of its turn.

    \begin{DndMonsterLegendaryActions}
      \DndMonsterLegendaryAction{Shaky Ground}{A 20 feet square becomes difficult terrain until the start of the elemental's next turn.}
      \DndMonsterLegendaryAction{Rumble}{Each creature within it's lair except the elemental must make a DC 12 dexterity saving throw or gain the prone condition.}
    \end{DndMonsterLegendaryActions}
  \end{multicols}
\end{DndMonster}
\begin{DndMonster}[float=!b]{Goblin}
  \DndMonsterType{Small humanoid (goblinoid), neutral evil}

  % If you want to use commas in the key values, enclose the values in braces.
  \DndMonsterBasics[
    armor-class = {15 (Leather Armor, Shield)},
    hit-points  = {\DndDice{2d6}},
    speed       = {30 ft.},
  ]

  \DndMonsterAbilityScores[
    str = 8,
    dex = 14,
    con = 10,
    int = 10,
    wis = 8,
    cha = 8,
  ]

  \DndMonsterDetails[
    skills = {Stealth +6},
    senses = {Darkvision 60 Ft., passive Perception 9},
    languages = {Common, Goblin},
    challenge = 1/4,
  ]
  % Traits
  \DndMonsterAction{Nimble Escape}
  The goblin can take the Disengage or Hide action as a bonus action on each of its turns.

  \DndMonsterSection{Actions}

  \DndMonsterAttack[
    name=Scimitar,
    distance=melee,
    type=weapon,
    mod=+4,
    reach=5,
    targets=one target,
    dmg=\DndDice{1d6+2},
    dmg-type=slashing,
  ]

  \DndMonsterAttack[
    name=Shortbow,
    distance=ranged,
    range=80/320,
    type=weapon,
    mod=+4,
    reach=5,
    targets=one target,
    dmg=\DndDice{1d6+2},
    dmg-type=piercing,
  ]
\end{DndMonster}

\subsubsection{Humans}
There are a few humans in this one-shot. None of those is expected to be
fought. If the players attempt this, fade to black as they are subdued by
the city guard or something.
\end{document}